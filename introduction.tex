\part{Introduction}
\section{Preface}
Thanks to the rapid development of experimental techniques at the cellular level, over the last century there was a tremendous progress in the fields of human physiology, immunology and therapeutics. The breakthrough discoveries were based mainly on the analysis of biochemistry (either intercellular or intracellular molecular pathways).
Mechanical properties of investigated objects were often either completely neglected, or strongly underestimated. However, all living cells are exposed to the action of external forces: fluid-mediated (eg. blood flow) or structure-mediated (eg. body weight strain in bones). The process of converting external mechanical stimuli into biochemical signals (and in turn into physiological responses) is called mechanotransduction. 
\newline 
The most appealing discovery, which evidences for the importance of mechanical properties, refers to the fate of stem cells. In 2006 Engler \emph{et al.} identified a new factor capable to regulate the fate of stem cells: the elasticity of the microenvironment (matrix). Namely, by changing the elastic properties of the substrate, stem cells could be directed towards muscle, bone or even neuronal lineages \cite{Even-Ram2006}. On the grounds of the recent discoveries, the new field of science have emerged: \emph{mechanobiology}. It is targeted to study the mechanotransduction processes at the level of tissues and cells, and the way it influences the development, physiology and diseases.
One may wonder what is the range of forces capable to elicit a cellular response. It is reasonable to assume, that the effect of force should exceed the energy of thermal fluctuations. At $37\degree C$ the thermal energy, $kT$, is about $4\;pN\cdot nm$. Considering the conformational changes in peptide-based molecular transducers have a characteristic length scale within the range of $1 - 10\;nm$, then it would correspond to the force of $0.4-4\;pN$. Interestingly, Finner \emph{et al.} (1994) have estimated, that a single myosin molecule, which drives contractility action and thus can induce cell signaling, is capable to produce a force $3-4\;pN$ \cite{Silberberg2008}. Therefore, it turns out that mechanotransduction is induced by forces only slightly higher than thermal fluctuations, \emph{i.e.} within the range of $10^0-10^1$ piconewtons. Obviously, the sensitivity and concentration of force transducers strongly depends on the type of cell and its location in human body.
\newline

The endothelium is formed by the monolayer of cells lining the lumen of all blood vessels in human body. Endothelial cells (ECs) serves as a barrier between blood and the rest of the system. Thus, its' physiology is affected by numerous biochemical factors. The vascular system is a highly dynamic structure, with component-rich blood constantly circulating in the pace of heart beats, which is followed by vessel vasodynamics. Therefore, a crucial input to cardiovascular physiology is provided by mechanobiology of endothelium, which in turn may be considered in terms of two separate classes: cellular mechanics and external mechanical stimuli.

the mechanobiology of endothelium has a curial impact on cardiovascular physiology. The endothelial nanomechanics can be considered in ter

Moreover, due to its location, it is directly exposed to mechanical stimuli.  

Physiological relevance of endothelial cells nanomechanics

\newline

Techniques to mechanically probing the cell

\newline

Cele pracy i struktura pracy

\newline

This work was supported by the European Union from the resources of the European Regional Development Fund under the Innovative Economy Programme (grant coordinated by JCET-UJ, No POIG.01.01.02-00- 69/09). Part of this project (tip-induced mechanotransduction) has been also supported by the grant of the Polish Ministry of Science and Higher Education number 7150/E-338/M/2013.


\section{Cellular structures determining mechanical properties of cells}