% ------------------------------------------------------------------------
% USTAWIENIA
% ------------------------------------------------------------------------

% ------------------------------------------------------------------------
%   Kropki po numerach sekcji, podsekcji, itd.
%   Np. 1.2. Tytu� podrozdzia�u
% ------------------------------------------------------------------------
\makeatletter
    \def\numberline#1{\hb@xt@\@tempdima{#1.\hfil}}                      %kropki w spisie tre�ci
    \renewcommand*\@seccntformat[1]{\csname the#1\endcsname.\enspace}   %kropki w tre�ci dokumentu
\makeatother

% ------------------------------------------------------------------------
%   Numeracja r�wna�, rysunk�w i tabel
%   Np.: (1.2), gdzie:
%   1 - numer sekcji, 2 - numer r�wnania, rysunku, tabeli
%   Uwaga og�lna: o otoczeniu figure ma by� najpierw \caption{}, potem \label{}, inaczej odno�nik nie dzia�a!
% ------------------------------------------------------------------------
\makeatletter
    \@addtoreset{equation}{section} %resetuje licznik po rozpocz�ciu nowej sekcji
    \renewcommand{\theequation}{{\thesection}.\@arabic\c@equation} %dodaje kropk�

    \@addtoreset{figure}{section}
    \renewcommand{\thefigure}{{\thesection}.\@arabic\c@figure}

    \@addtoreset{table}{section}
    \renewcommand{\thetable}{{\thesection}.\@arabic\c@table}

    \@addtoreset{section}{part} %resetuje licznik po rozpocz�ciu nowego partu
\makeatother

% ------------------------------------------------------------------------
% Tablica
% ------------------------------------------------------------------------
\newenvironment{tablica}[3]
{
    \begin{table}[!tb]
    \centering
    \caption[#1]{#2}
    \vskip 9pt
    #3
}{
    \end{table}
}

% ------------------------------------------------------------------------
% Dostosowanie wygl�du pozycji listy \todos, np. zamiast 'p.' jest 'str.'
% ------------------------------------------------------------------------
\renewcommand{\todoitem}[2]{%
    \item \label{todo:\thetodo}%
    \ifx#1\todomark%
        \else\textbf{#1 }%
    \fi%
    (str.~\pageref{todopage:\thetodo})\ #2}
\renewcommand{\todoname}{Do zrobienia...}
\renewcommand{\todomark}{~uzupe�ni�}

% ------------------------------------------------------------------------
% Definicje
% ------------------------------------------------------------------------
\def\nonumsection#1{%
    \section*{#1}%
    \addcontentsline{toc}{section}{#1}%
    }
\def\nonumsubsection#1{%
    \subsection*{#1}%
    \addcontentsline{toc}{subsection}{#1}%
    }
\reversemarginpar %umieszcza notki po lewej stronie, czyli tam gdzie jest wi�cej miejsca
\def\notka#1{%
    \marginpar{\footnotesize{#1}}%
    }
\def\mathcal#1{%
    \mathscr{#1}%
    }
\newcommand{\atp}{ATP/EMTP} % Inaczej: \def\atp{ATP/EMTP}

% ------------------------------------------------------------------------
% Inne
% ------------------------------------------------------------------------
\frenchspacing                      %nie pami�tam co to jest, ale u�ywam
%\flushbottom                       %nie pami�tam co to jest, ale nie u�ywam
%\raggedbottom                      %nie pami�tam co to jest, ale nie u�ywam
\hyphenation{ATP/-EMTP}             %dzielenie wyrazu w ��danym miejscu
\setlength{\parskip}{3pt}           %odst�p pomi�dzy akapitami
\linespread{1.2}                    %odst�p pomi�dzy liniami (interlinia)
\setcounter{tocdepth}{4}            %uwzgl�dnianie w spisie tre�ci czterech poziom�w sekcji
\setcounter{secnumdepth}{4}         %numerowanie do czwartego poziomu sekcji w��cznie
\titleformat{\paragraph}[hang]      %wygl�d nag��wk�w
{\normalfont\sffamily\bfseries}{\theparagraph}{1em}{}
%\definecolor{niebieski}{rgb}{0.0,0.0,0.5}
